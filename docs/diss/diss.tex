% Template for a Computer Science Tripos Part II project dissertation
\documentclass[12pt,a4paper,twoside,openright]{report}
\usepackage[pdfborder={0 0 0}]{hyperref}    % turns references into hyperlinks
\usepackage[margin=25mm]{geometry}  % adjusts page layout
\usepackage{graphicx}  % allows inclusion of PDF, PNG and JPG images
\usepackage{verbatim}
\usepackage{docmute}   % only needed to allow inclusion of proposal.tex
\usepackage{makecell}
\usepackage{amsmath}
\usepackage{listings}

\raggedbottom                           % try to avoid widows and orphans
\sloppy
\clubpenalty1000%
\widowpenalty1000%

\renewcommand{\baselinestretch}{1.1}    % adjust line spacing to make
                                        % more readable

\begin{document}

\bibliographystyle{plain}


%%%%%%%%%%%%%%%%%%%%%%%%%%%%%%%%%%%%%%%%%%%%%%%%%%%%%%%%%%%%%%%%%%%%%%%%
% Title


\pagestyle{empty}

\rightline{\LARGE \textbf{Alex Bostock}}

\vspace*{60mm}
\begin{center}
\Huge
\textbf{A Comparison of Consistency Models in Distributed Database Systems} \\[5mm]
Computer Science Tripos -- Part II \\[5mm]
Churchill College \\[5mm]
\today  % today's date
\end{center}

%%%%%%%%%%%%%%%%%%%%%%%%%%%%%%%%%%%%%%%%%%%%%%%%%%%%%%%%%%%%%%%%%%%%%%%%%%%%%%
% Proforma, table of contents and list of figures

\pagestyle{plain}

\chapter*{Proforma}

{\large
\begin{tabular}{ll}
Name:               & \bf Alex Bostock                          \\
College:            & \bf Churchill College                     \\
Project Title:      & \makecell[tl]{\bf A Comparison of Consistency Models in \\ \bf Distributed Database Systems}\\
Examination:        & \bf Computer Science Tripos -- Part II, July 2019  \\
Word Count:         & \bf 0  \\
Project Originator: & A. T. Bostock                    \\
Supervisor:         & Dr J. K. Fawcett                    \\ 
\end{tabular}
}
\stepcounter{footnote}


\section*{Original Aims of the Project}

TODO

\section*{Work Completed}

TODO

\section*{Special Difficulties}

TODO
 
\newpage
\section*{Declaration}

I, [Name] of [College], being a candidate for Part II of the Computer
Science Tripos [or the Diploma in Computer Science], hereby declare
that this dissertation and the work described in it are my own work,
unaided except as may be specified below, and that the dissertation
does not contain material that has already been used to any substantial
extent for a comparable purpose.

\bigskip
\leftline{Signed [signature]}

\medskip
\leftline{Date [date]}

\tableofcontents

\listoffigures

%%%%%%%%%%%%%%%%%%%%%%%%%%%%%%%%%%%%%%%%%%%%%%%%%%%%%%%%%%%%%%%%%%%%%%%
% now for the chapters

\pagestyle{headings}

\chapter{Introduction}

% TODO: Make it clear in the first paragraph what your project is about \& how well you’ve done it

\section{The Need for Distributed Databases}

Web apps are big business. For example, Facebook has over 1.5 billion daily active users \cite{facebook}. There are $86,400$ seconds in a day, so the peak number of users active within a second must be at least $1,500,000,000/86,400 > 17,000$.
Big web apps like this need big databases.

% TODO: tidy up first paragraph, and mention Facebook's value

Rather than a centralised database, we want a distributed one, which has several advantages \cite{lake2013}:

\begin{itemize}
\item
Centralised databases have a single point of failure; with a database running on a single machine, the database is unavailable if that machine fails. Even supposing we have a very reliable machine, we still have to have some downtime to upgrade that machine's OS. Distributed databases systems can be made to remain available after several individual nodes have failed \cite{bacon2003operating}.

\item
Centralised databases are entirely dependant on a single machine's network connection. We don't want a local network failure causing the entire database to become unavailable. This is another single point of failure, independent of the actual machine.

\item
In the case of media failure on the database machine, we will probably lose data. With distributed systems, we can keep multiple redundant copies of the same data, so data loss is much less likely.

\item
Distributed systems are easier and cheaper to scale. Centralised systems can only be scaled vertically: by buying more powerful hardware. Beyond a point, the cost of more powerful hardware is unreasonable. Distributed systems can be scaled horizontal: by adding more machines. Many weaker machines are cheaper than a single supercomputer.

\item
Distributed systems may be able to provide lower latency. If users are distributed around the world, a request may be served with lower latency by querying a local data centre. This is not possible with a centralised system (mean Sa\~o Paulo to Singapore RTT is 362.8ms, only 22.5ms California to Oregon \cite{bailis2013highly}).

\end{itemize}

\section{Consistency and Availability Trade-Offs}

Distributed databases have a trade-off of consistency against availability. This was proposed as the CAP theorem, which says a system cannot have all 3 of consistency, availability and partition tolerance \cite{brewer}. The exact definitions of consistency and availability were not precisely defined.

A partition tolerant system is one which continues to function correctly when the network is not connected. In a partition scenario, all nodes in the network are split into 2 or more disjoint sets; each node can send messages to other nodes in the same set, but not to those in other sets (on the other side of a partition). Given the decentralised nature of the Internet, most web applications require partition tolerance \cite{hale_2010}. % TODO: make sure Spanner is mentioned somewhere

One definition of consistency, also known as strong consistency, is linearisability. This means that all transactions are serialisable and, if some transaction $x$ happens before another transaction $y$, then $y$ sees all effects of $x$. Serialisability means that each transaction appears to be executed instantaneously so, at any particular point in time, every transaction's effects can either be completely seen, or not seen at all \cite{herlihy1990linearizability}.

The strictest definition of availability says, as long as at least one database node is reachable by a caller, database transactions can be completed. The CAP theorem has been proven, based on this definition of availability and consistency defined as linearisability (strong consistency) \cite{gilbert}.

There are many other types of consistency, most notably eventual consistency. An eventually consistent system may not appear consistent but, after some indefinite period of time without updates, will converge to a strongly consistent state. This can be combined with a variety of additional guarantees, each useful for different applications \cite{vogels_2008}.

In practice, no real world system has perfect availability, as defined in the CAP theorem proof. Instead, we want to measure availability. A useful metric is yield: the proportion of transactions which are successful \cite{fox1999harvest}. Ideally, a system should provide yield very close to $100\%$.

Although availability is helpful, it isn't the only useful metric. We also need to consider latency. A system could have perfect availability but sometimes take hours to respond to a request, which is not useful in many applications. An alternative to the CAP theorem has been proposed: PACELC. This says that, in the presence of partitions, there is a trade-off of availability against consistency; otherwise, there is a trade-off of latency against consistency \cite{abadi2012consistency}.

\section{Databases for Social Networking Services}

My project explores what availability and latency can be achieved while providing different consistency guarantees, particularly considering the requirements of large-scale social networking services. Applications like this have several important characteristics.

\begin{itemize}
\item
Databases need to process high volumes of ``read-mostly'' database transactions \cite{fox1999harvest} \cite{nunemaker}.

\item
Availability is important, since better service means more revenue, especially with business models based on advertising. Latency is also important, for similar reasons (``for every additional second a page takes to load, 10 per cent of users leave'' \cite{clark_2018}).

\item
Different services within the application require different consistency guarantees. For example, when serving a news feed to a user, high availability is much more important than strong consistency, but consistency may be much more important when determining whether a particular user has permission to see a post.

\end{itemize}

\chapter{Preparation}
% 26% (with intro)

% TODO: Compare to other systems (Chubby \cite{27897}, Spanner \cite{45855}, BigTable \cite{chang2008bigtable}, Dynamo \cite{decandia2007dynamo}, etc.)

\section{Overview}

% TODO

\section{Tools Used}

% TODO

\section{Starting Point}

% TODO

\section{Public Interface}

Previously, nearly all database systems were relational database \cite{lake2013}. These were characterised by reasonably complex data models, allowing complex structured queries with SQL, and minimising the amount of redundant data stored. More recently, this different systems have become more popular, preferring simpler data models so that systems are more scalable \cite{bailis2013highly}.

Given this, I built a distributed key-value store. This has a very simple data structure: each record is composed of a key and a value. The store is indexed by keys, and there can be at most one record present with any particular key. Essentially, the database is a persistent, distributed hash table. The public interface has 2 methods:

\begin{itemize}
\item
\verb|put(key, value)| either stores the given key and value in the system, or returns an error. This can also be used to delete a value, by storing the null value. If the transaction was completed successfully, put also returns a unique timestamp associated with the value stored.

\item
\verb|get(key)| either returns a value associated with the given key and that value's timestamp, or returns an error response.

\end{itemize}

The put method has at-most-once semantics. A transaction may or may not be completed, but will not be executed more than once. If a client receives no response to a request, there is no way of knowing whether or not the transaction was completed; the node coordinating the transaction may have failed before executed the transaction, or have failed immediately after executing it, but before replying. This limitation is reflected in the set of put response values: success, failure and unknown.

The get method is idempotent: it never modifies the data store. The only goal of a get is to return a value to the client. If no response is received, the transaction failed from the client's point of view. This means get has only two response values: success and failure.

Note that a successful get request may return a null value, indicating that the requested key is not present in the database. Importantly, the null response (value not present) is different from the failure response (value not found).

All keys and values can be byte arrays of any size. The database does not inspect their values, so the type and structure of data stored is irrelevant.

Put requests do not read any current value stored, except for timestamps. This means the system cannot be reliably used for larger atomic operations (without using some external locking system). % TODO: implement and change here

% TODO: Semantics of overwriting old values, alternatives (larger atomic transactions), usefulness to applications

\section{Consistency Guarantees}

\subsection*{Strong Consistency}

The strongest consistency guarantee is strong consistency, which is defined as linearisability. Each transaction takes has a non-zero latency, so can be associated with 2 timestamps: a start time and end time. The system should appear as if every transaction occurs instantaneously at a unique timestamp, and that timestamp is between the transaction's start and end times \cite{herlihy1990linearizability}.

Note that there is no universal concept of time \cite{bacon2003operating}, but guarantees based on universal time are useful. If a single client completes several transactions sequentially, they should appear to be executed sequentially. In a more complicated scenario, we may have one client $x$ which completes a transaction, then sends a message to another client $y$. Having received the message, $y$ then completes a transaction. With a global concept of time, since the message cannot be received before it is send, we may want to ensure that $y$'s transaction does appear to be executed after $x$'s.

Strong consistency is sufficient for any application, but is not necessary for every application. Since the CAP theorem shows that a strongly consistent, partition-tolerant system must sacrifice availability \cite{gilbert}, we need to consider weaker consistency guarantees which may still be useful for some applications.

\subsection*{Eventual Consistency}

An eventually consistent system does not provide strong consistency. If a write then a read are executed sequentially, there is no guarantee that the read will return the value just written. What eventual consistency does guarantee is, after some indefinite amount of time passes without any updates to the datastore, the system will reach a strongly consistent state \cite{vogels_2008}. Note that the time required to reach consistency is not defined.

Eventual consistency is useful for some applications. For example, when a user creates a social media post, it is not essential that the post is immediately visible to users everywhere in the world. after some time, the post should be visible everywhere. Also note that reaching that state in a very short time is desirable but not essential. It may be the case that a 1 second delay before a post is completely available is acceptable if the database throughput can be increased by a factor of 10, for example.

\subsection*{Session Guarantees}

While strong consistency is not required for all applications, many applications require stronger guarantees than eventual consistency. One type of guarantee which can be provided is a session guarantee.

``A session is an abstraction for the sequence of read and write operations performed during the execution of an application.'' \cite{terry1994} A client can request additional guarantees which are valid within the scope of a session. Different guarantees can be useful for different applications.

\begin{description}
\item{Monotonic Reads} means that each read within a session is at least as up-to-date as the previous one. If there are two sequential reads, the second one will always return the same value as the first, or return a more recent value. This guarantee can prevent oscillating behaviour from a client's point of view. For example, once a social media user has seen a post, that user will always be able to see that post unless the post is subsequently deleted (as long as the service is available).

\item{Monotonic Writes} means that writes within a session are applied sequentially, in the correct order. Suppose a social media post is edited several times, with each edit replacing the previous version of the post. With monotonic writes, the most recent version is guaranteed to be the version which persists in the end.

\item{Read Your Writes} (also known as Read My Writes) means that each read within a session see the effects of all previous writes in the same session. This may be useful for an authentication system; if a user changes their password then tries to sign in with the new password, an eventually consistent system may reject the password. With the Read Your Writes guarantee, this scenario would not occur.

\end{description}

\subsection*{Other Guarantees}

Other guarantees can be provided independently of sessions \cite{terry2013}.

\begin{description}
\item{Consistent Prefix} means that every read sees the effects of an ordered sequence of writes beginning with the first write. The length of the sequence is not defined. This means that each read returns a snapshot of a previous state of the database. % TODO: an example

\item{Bounded staleness} means that the value returned by a return is either the most recent value, or was the recently value at some point in a defined time frame. Where eventual consistency does not define the time to reach a consistent state, bounded staleness might guarantee that the system will reach a consistent state (if it is available) in at most 1 hour, for example. Suppose a social networking service needs to delete a video for legal reasons; with bounded staleness, the system could guarantee that the video was completely removed within 5 minutes of the transaction. While strongly consistent may be more suitable, a bounded delay may be acceptable to avoid the associated performance costs.

\item{}
Any of the session guarantees listed above can also be provided independently of sessions, effectively defining a global session which includes all transactions.

\end{description}

\section{Quorum Assembly}

My system uses quorum assembly to enforce consistency. For strong consistency, we use a strict quorum system.

Assume the system has $n$ database nodes. Define 2 parameters $V_R$ and $V_W$, which satisfy the following conditions:

$$V_W > n / 2$$

$$V_R + V_W > n$$

$V_W$ is the number of nodes required for a write transaction. The first condition means that every write sees every other write: each write involves more than $n/2$ nodes chosen from the same set of $n$, so there must be at least 1 node involved in both transactions. This ensures that every write to a particular key will have a unique timestamp, so there is a unique ordering of write transactions.

The first condition also means that, once a write has been committed, there are at more than $n/2$ copies of the value written, so no data will be lost in the event of total failure of any $n/2$ database nodes.

$V_R$ is the number of nodes required for a read transaction. The second condition means that every read sees every write. Since conflicting values for the same key must have unique timestamps, every read will always return the most recent value, satisfying the conditions for strong consistency.

\section{Sloppy Quorum}

If strong consistency is not required, we can use a quorum system with more relaxed constraints. We can achieve different performance and different consistency guarantees by imposing different constraints. Consider removing each from a strict quorum system.

With both $V_W > n/2$ and $V_R + V_W > n$, the system is strongly consistent, as described in the previous section.

With only $V_W > n/2$, the system guarantees monotonic writes (globally rather than within a session). This means that there is a total ordering on write transactions. By removing the constraint on $V_R$, any particular read is not guaranteed to see any particular write. If every write eventually reaches at least $n - V_W + 1$ nodes, the system is eventually consistent. This can be implemented easily; after a write transaction is atomically committed to $V_W$ replicas, it can be lazily propagated to other nodes. Given the total ordering on writes, every write can have a unique timestamp, so the system can easily avoid overwriting newer values with older ones, so each replica becomes monotonically more up-to-date over time.

With only $V_R + V_W > n$, the system is optimised for write operations. This may be useful for storing debug logs; writing to logs can be a frequent operation, without a big performance cost to the application. If logs are read infrequently, high latency and poor availability for reads is acceptable. Also note that, since $V_W$ can take any value, the likelihood of data loss due to individual node failures is increased. In the most extreme case, $V_W = 1$, so data loss may occur if a single node suffers media failure. This may be acceptable for debug logs, but most applications need better durability.

With no constraints on $V_R$ and $V_W$, the system provides very few guarantees. Writes can occur concurrently, so the quorum system cannot provide a total ordering of writes. Although it guarantees no consistency, an unconstrained quorum system can provide the best availability. In the most extreme case, $V_R = V_W = 1$, so the system is available as long as a single node is reachable.

% TODO: mention Dynamo

With variations on the basic quorum system, we can build systems with many diffferent consistency guarantees.

\begin{table}[h!]
\centering
\begin{tabular}{| p{0.25\linewidth} | p{0.6\linewidth} |}
\hline
Strong Consistency & $V_R + V_W > n$ and $V_W > n/2$\\
\hline
Eventual Consistency & Background write propagation\\
\hline
Monotonic Reads & Always read from the same node\\
\hline
Monotonic Writes & Always write to the same node (with original node coordinating write), or $V_W > n$ (enforcing globally rather than session)\\
\hline
Read Your Writes & Both monotonic reads and monotonic writes\\
\hline
Consistent Prefix & $V_W > n/2$\\
\hline
Bounded Staleness & Probabilistically from convergence time, or add a mechanism to restrict availability based on time since last contacted each other node.\\
\hline
\end{tabular}
\end{table}

% TODO: make this better
I implemented a strict quorum system as a control. I also implemented a system with the constraint $V_W > n/2$, which provides a useful set of consistency guarantees.

I implemented two systems. The first uses strict quorum to provide strong consistency.

The second uses sloppy quorum, enforcing $V_W > n/2$ only, providing consistent prefix and monotonic writes, and uses a background write propagation mechanism to provide eventual consistency. In addition, monotonic reads can be achieved within a session by sending all read requests to the same node. A read in the sloppy quorum system could read from more than $V_R$ nodes, so a client could choose between strongly consistent reads and eventually consistent ones.

\section{Concurrent Writes}

Both of the systems I implemented have the constraint $V_W > n/2$. This prevents 2 write operations from executing concurrently, which enforces a total ordering on all write operations. This provides linearisability, ensuring each write appears to be executed instantaneously at a unique timestamp. The cost of this is in availability and latency: every write transaction needs to reach a majority of replicas.

In a system without this constraint, writes may occur concurrently. This means that there may be several different values stored with the same key and the same timestamp. There are several different ways to deal with this.

A system could use some deterministic method of breaking ties. For example, with every database node having a unique ID, always take choose the value from the node with the highest ID. This is deterministic across the system, and provides serialisability. This means that there is a total ordering on writes, but this is not always related to the actual order of writes. For an application, this creates a race condition; any time there are too processes concurrently writing to the same value, the value which persists is not defined.

A more useful tie break might be based on the client's ID. For example, a social media post may be editable only by the user which created it, or a moderator. A database could always prefer writes by a moderator. This is deterministic, and simple and helpful to an application developer.

In order to provide linearisability without locking, a system needs to use synchronised clocks. Using GPS clocks and very low latency networking within a data centre, clocks in a distributed system can be synchronised within a very small margin. Using this, every transaction can have a precise UTC timestamp, with small, known error margins. This technique is used by Google's Spanner \cite{45855}. This also has the benefit of allowing consistent snapshots. Without locking an entire database, it is possible to query the state of the whole system at a particular time in the past.

\section{System Limitations}

My design has a number of important limitations, which are summarised here.

\begin{itemize}
\item
The set of database nodes is fixed. Additional nodes cannot be added to the system, and existing ones cannot be removed. In practice, the ability to change the set of nodes is useful for incrementally scaling a system. Amazon's Dynamo achieves this by storing each value on a subset of the replicas in a cluster, using consistent hashing to limit the number of values which need to be relocated each time the number of replicas changes \cite{decandia2007dynamo}. Incremental scaling is outside the scope of my project.

\item
% TODO: Rewrite this
Nodes cannot fail permanently. In 2PC, if a node fails, we need to wait for that node to become available again before allowing further writes, to ensure that the system remains consistent.

Mitigation: just require manual intervention. With fine-grained locking, most of the database can remain available throughout.

A small number of keys having to be manually un-jammed is not a huge problem, assuming they fail independently. See application decomposition in \cite{fox1999harvest}.

% TODO: Justify not just adding a timeout \cite{imbriaco_2012}

\item
% TODO: Rewrite this
If any database node's persistent store is irreversibly corrupted on failure, then consistency guarantees may be violated.
% TODO: justify based on RAID, Reed-Solomon

\end{itemize}

\chapter{Implementation}
% 40%

% TODO: polish all the writing in this chapter

\section{Data Storage}

Each database node in the system needs its own key-value store. This will be built as a module independently of all distributed components of the system, and can be tested as a stand-alone module. Since the main focus of this project is distributed systems, and I expect the performance bottleneck to be network latency, this module is designed preferring simplicity over optimisation. As an extension, I may explore more efficient data structures for storage.

The data store is built on top of a Unix-like file system. The particular file system does not matter, provided modification to a file descriptor (eg. renaming a file) is an atomic operation.

Data will be stored as binary files. The interface says that keys and values are byte arrays, which may not contain the null character. Keys and values will be written to disk as binary, using the null character (\verb|\0|) as a delimiter. The content of any valid file will match the format:

\verb|(Key \0 Value \0)*|

Also, a file must contain at most one occurrence of any key. Given this, any file can be uniquely deserialised to a map from byte array to byte array.

There is no requirement for any ordering of keys in a file, so the most efficient way to lookup a key is a linear search, which will take $O(n)$ time when there are $n$ key-value pairs stored. To mitigate this, data will be split between multiple files, based on a hash of the key.

A simple hash function could use the prefix of the key: the first 2 bytes, for example. This would divide data uniformly if keys are random, but may group many values together if their keys are similar (which is entirely determined by the application programmer).

Instead, I will use MD5. While not cryptographically secure, MD5 is a relatively fast hashing algorithm. Each file in the data store will correspond to the first 2 bytes of an MD5 hash. This should divide data uniformly, assuming that keys are not chosen with intent to maximise the number of collisions. The name of each file will be the 2 byte prefix, encoding as a hexadecimal string to avoid use of invalid characters in filenames. This will also not require the filenames to be case sensitive in the underlying file system.

The most important method for the data store to provide is atomic writes. This is achieved by writing to a shadow data structure.

My system does not allow concurrent write transactions, and each write modifies 1 key-value pair. This means that each transaction modifies one data file, and all writes are sequential. The write procedure for modifying a file $f$ is:

\begin{itemize}
\item
Create a new file $g$.

\item
Write data to $g$, which will be the contents of $f$, modified based on the transaction's parameters.

\item
Wait for a commit instruction.

\item
If a commit instruction is received, change the file descriptor $f$ to point at $g$ (\verb|mv g f|).

\item
If, instead, a rollback instruction is received, delete $g$.

\end{itemize}

% TODO: rewrite: Hash table on disk

%  - But many nodes sharing one disk won't work, so

%  - Trie in memory, with added delay

\section{Quorum Assembly}

The main method for ensuring consistency in my system is quorum assembly. For a strongly consistent system, strict quorum assembly is needed. For eventual consistency, this will be modified to use a sloppy quorum method.

Given a system of $n$ nodes, we first have to define two values $V_R$ and $V_W$, which are the sizes of read and write quorums, respectively. At least $V_R$ nodes must be involved in every read transaction, and at least $V_W$ nodes must be involved in every write. In a strict quorum system, we also enforce the following constraints:

$$V_W > n / 2$$

$$V_R + V_W > n$$

The first constraint means that every value written to the database must be written to a majority of database nodes. The second constraint means that, for any read quorum, at least one node in the quorum will have the most recent version of each value stored. Where multiple conflicting values exist for the same key, we need Lamport timestamps to identify the most recent value.

Lamport timestamps are monotonically increasing integers, which indicate the order in which some events occur. If some event $A$'s timestamp is smaller than another event $B$'s, then $A$ happened before $B$.

In my system, each node stores a Lamport timestamp for each database key. Where different nodes have conflicting values for the same key, the one with the greater clock value is most recent.

In order to ensure Lamport clock values do not overflow, and thus to ensure they are monotonically increasing, my system will use 64 bit unsigned integers for Lamport clocks. A clock value is incremented each time the associated key-value pair is overwritten. Assuming a high throughput, with 1 transaction every 10 milliseconds writing to the same key, the time before the Lamport clock overflows will be $2^{64} \times 10 \cdot 10^{-3} = 1.84 \cdot 10^{16} \text{seconds}$. This is more than 5 billion years, so 64 bit timestamps should be adequate.

Building on the basic data store interface, Lamport clock values will be stored as part of the value. At the lower level, each value is actually the concatenation of a Lamport timestamp and the data being stored. In the case when a value is deleted, the concatenation of a timestamp and an empty byte array will be stored. This indicates that a value has been deleted from this node, which will be necessary for reads (for example, when a read quorum contains some nodes on which a value has been deleted, and some which still have a stale value), but does not violate the requirement of not storing null values in the low-level store.

The procedure for a read transaction is:

\begin{itemize}
\item
A node receives a read request from the client. This node shall be the coordinator.

\item
Assemble a quorum of at least $V_R$ nodes, and obtain a lock on each node in the quorum.

\item
The coordinator queries each node in the quorum with the given key.

\item
Each node returns the value it has stored, and the Lamport clock associated with the key, or says it has no value stored for the key.

\item
The coordinator identifies the value with the highest Lamport timestamp, and returns that value as the result.

\item
The coordinator releases all locks.

\end{itemize}

Note that this may reach a deadlock condition. Suppose there are multiple concurrent transactions, concurrently acquiring locks on nodes. This may reach a condition where every node is locked, but no coordinator has assembled a complete quorum. If latency was not an issue, we could avoid this by always acquiring locks in the same order, ordered by node IDs. This would require $O(V_W)$ RTTs. Since $V_W > n / 2$, this would add significant latency with a large number of nodes (eg. $RTT = 20$ ms, $V_W = 500$ means acquiring locks would take $2$ seconds). Note that on average half of the nodes involved are unavailable throughout those 2 seconds, so this would have a significant effect on availability.

There will also be a problem if the coordinator fails before it releases locks. This would leave nodes permanently locked, and effectively useless.

To fix both of these problems, each node should set a timer when its lock is acquired, and reset that timer each time it responds to a request from the coordinator. If the timer runs out, it should release its lock. If it subsequently receives a request from the coordinator, it should return an error. The coordinator, if it receives an error, should abort and retry the transaction.

Note that, since every node in a read quorum is locked, and given the constraint $V_R + V_W > n$, there cannot be concurrent read and write transactions. This imposes a happens-before relationship between each read transaction and every write transaction, so ensures that the value read is consistent. There may be concurrent read transactions; this has no effect on consistency, and will allow greater throughput.

The procedure for write transactions is similar to that for reads, but requires additional work to ensure atomicity. When a value is written, it must be either written to a full write quorum, or not written at all. This is achieved using an atomic commit protocol, 2 phase commit (2PC).

The write procedure is:

\begin{itemize}
\item
A node receives a write request from the client. This node shall be the coordinator.

\item
Assemble a quorum of at least $V_W$ nodes, and obtain a lock on each node in the quorum.

\item
The coordinator queries the existing value for the required key from each node in the quorum. Note that this is essentially a read transaction before a write transaction, which is why strict quorum is sometimes describe as ``read your own writes''

\item
Each node returns the value it has stored for the given key, and the Lamport clock associated with that key.

\item
The coordinator identifies the most recent timestamp associated with the key. Note that, given the constraint $V_W > n / 2$, this must be the most recent write for the key. The coordinator increments the greatest timestamp to get the timestamp for the new write.

\item
The coordinator initiates 2PC with all the nodes in the quorum, to write the new value and the new Lamport timestamp.

\item
The coordinator releases all locks.

\end{itemize}

The same deadlock condition as with read transactions applies here, so we will use the same timer system for pre-emption.

\section{Two-Phase Commit (2PC)}

In 2PC, the coordinator begins by sending the transaction to each node in the quorum, as well as sending a list of nodes involved in the transaction.

Each node should then execute the transaction, writing the new value to persistent storage but not yet committing. It should save a new vector clock value with the new value, which is the merge of its existing vector for the key, and the vector clock from the coordinator. It should then respond to the coordinator with a yes vote, and a copy of its new vector clock for the key. In case of any failure, it should respond with a no vote, and can cancel the transaction.

If the coordinator receives a yes vote from every node, it should:

\begin{itemize}
\item
Send a commit message to each node.

\item
Wait for an acknowledgement from each node.

\item
When it receives an acknowledgement from every node, commit the transaction to its own disk.

\end{itemize}

In the case that the coordinator does not receive an acknowledgement from every node, it should repeatedly send commit messages until it does so. This is one drawback of 2PC: it is a blocking protocol. Until all nodes involved have acknowledged, a write transaction cannot be completed. Note that all nodes involved are locked throughout this time; given that concurrent writes are not allowed, this will affect the availability of the system, but it does provide consistency.

If the coordinator doesn't receive a yes vote from every node (either at least one no vote, or a response times out), it should send a rollback message to all nodes.

In case of the coordinator failing, all other nodes in the transaction will timeout. In this case, since they all know the identifiers of all nodes involved in the transaction, the node with the lowest identifier becomes a new coordinator. The new coordinator queries all other nodes for their votes. In case any node voted no, it should be aborted. Otherwise, the transaction can be completed. When the coordinator recovers, it can determine from the other nodes whether or not the transaction was committed.

2PC is tolerant of most failure scenarios.

\begin{itemize}
\item
In the case of a node (or several nodes) other than the coordinator failing before sending a yes vote, the coordinator will abort the transaction.

\item
In the case of a node other than the coordinator failing after sending a yes vote, the failed transaction will be stored persistently on that node, so the node will be able to recover.

\item
In the case of the coordinator failing, all other nodes involved in the transaction can communicate to decide whether or not to abort. The transaction will be stored persistently on the coordinator node, so the coordinator can recover in the same way as any other node failing after voting yes.

\end{itemize}

A limitation of this is that it is not tolerant of data loss on failure, for example due to disk failure. This is a limitation of my system. In the real world, each node could store data on multiple disks in a RAID array, making the chance of data loss negligible. In case of total loss of database nodes, data could still be recovered provided at least half of the database nodes are unaffected, although at the cost of availability.

Another issue is the case of simultaneous failure of both the coordinator and another node. In this case, the remaining nodes could try to recover, but could not know whether the second failed node voted yes or no. This is a limitation of my system. As an extension, I may modify the system to use three-phase commit (3PC), which addresses this problem.

\section{Sloppy Quorum}

Sloppy quorum is a modification to the strict quorum system, which does not provide strong consistency guarantees. The procedure for transactions is the same as before, but we remove the constraint $V_R + V_W > n$.

This means that a read transaction may not return the most recent value for the requested key; it may return a stale value. The benefit is that more read transactions can occur concurrently, which will allow a greater throughput of transactions, and the system can remain available for read after a larger number of node failures: as long as at least $V_R$ nodes are available, read transactions can still occur.

In order to provide eventual consistency, we need to change the system more. In an eventually consistent system, if there are no write transactions, the whole system should reach a consistent state after some amount of time. This means that every read transaction should eventually return the most recently written value for each key. Having removed the constraint on $V_R$, this means that values should eventually be written to at least $n - V_R + 1$ nodes.

In order to achieve this, I will add an additional mechanism for sharing values between database nodes. Each write transaction has a coordinator, and that coordinator will be responsible for ultimately propagating the write to at least $n - V_R + 1$ nodes. My system assumes that the set of nodes is fixed, and there is no requirement for the time to reach consistency, so the coordinator can resume this task after failure and recovery without issue.

For each database key, each node should store an integer $x$ representing the number of nodes which have definitely received the value stored.  When a value is written by a transaction coordinated by another node, that value's $x$ can be set to $n$; although not true, the propagation of that value is another node's responsibility. When a node coordinates a write transaction, it should set $x$ to the number of nodes in the write quorum.

Each node then needs a background process to continually propagate values for which $x$ is smaller than $n - V_R + 1$. This could be done by initiating write transaction, but doing so would take much more time, and the consistency guarantee is not required.

Instead, a node can send a lightweight write message to other nodes. When a lightweight write message is received, a node should compare the given key, value and timestamp to what it has stored. If the value received is more recent than that stored, it should update its stored value and reply. Otherwise, it should reply with its more recent value and timestamp. When receiving a response from a lightweight write, a node can either increment $x$ for the relevant key, or store a more recent value in its store.

In case of failure of a lightweight write, there is no problem. The guarantees provided by full write transactions are still the same, and lightweight transactions can be retried later to eventually achieve consistency.

\section{Leadership Election}

With $n$ nodes, my system constrains the write quorum size $V_W$ with$V_W > n / 2$. This means that there can never be concurrent writes; if two nodes each simultaneously attempt to assemble a quorum for a write transaction, they cannot both succeed at the same time. This means there has to be a mechanism to avoid a deadlock condition. I implemented a timeout, which aborts a transaction if assembling a quorum takes too much time; one of the conditions for deadlock is no preemption, so a deadlock cannot occur.

While this recovers from the deadlock, usually both transactions fail in this case, and often a large number of nodes are locked until the transactions are aborted. This has a significant effect on the system's throughput. My system allows any database node to receive client requests and coordinate transactions. Since attempted concurrent writes are expensive, I built a mechanism to minimise the number of instances when this occurs. This is based on a ring election algorithm.

The idea is, since there cannot be concurrent writes, make a single node responsible for coordinating all write transactions. Rather than rely on a single node (and be unavailable for writes when the node is unreachable), a node is elected. Note that attempted concurrent writes affect performance rather than correctness, so my system can tolerate no consensus on the current leader (different nodes disagree on the current leader), although this should be avoided.

All nodes are considered part of a virtual ring. This is defined based on node IDs, since these are assigned sequentially. If there are $N$ nodes in total, each node $n$ is adjacent to nodes $(n + 1) \% N$ and $(n - 1) \% N$. There is a token message continually forwarded around the ring containing a list of node IDs. When each node receives the token, it does the following:

\begin{itemize}
\item
Send an acknowledge (ACK) message to the sender (the previous node in the ring).

\item
Add its own ID to the token, if not already present.

\item
Pick the current leader as the highest ID listed in the token, updating its local record of the leader ID.

\item
Forward the token to the next node in the ring.

\item
If no ACK is received after a timeout, remove the next node's ID from the token (if present), and forward the token to the node after next.

\end{itemize}

% TODO: check the code actually does this

This means that an unreachable node will be detected after at most the time taken for the token to traverse the ring, plus the timeout length. After another traversal of the ring, every node will be aware of the new leader.

The token may be lost, if a node fails before forwarding it. To fix this, any node which does not receive the token for some time should create a new token. Until a new token has traversed the ring, nodes will disagree on the current leader, but will reach consensus after one traversal (unless the token is lost again). This is acceptable for my system.

Since any node may create a token, the system may reach a state when there are many tokens. This is not a problem, but may cause problems if there are too many tokens in use, each using network capacity and processing time across the system. To fix this, each node has a rate-limiter. Each node enforces a minimum time between forwarding 2 tokens. Any tokens to be sent before the minimum time has elapsed are discarded, limiting the total number of tokens in use.

% TODO: some data to justify the benefit of this, compare to bully

\section{Thread Confinement}

My database system, like all distributed systems, uses a lot of concurrency, both between separate database nodes and clients, and within each node. Building concurrency systems requires a lot of work to ensure thread-safety.

My system is built in Go, which provides some helpful primitives for concurrent programming: goroutines and channels.

Goroutines are similar to threads. Rather than using OS-level scheduling, the Go runtime has its own scheduler. Goroutines, unlike normal threads, have dynamically-sized stacks.

Typically, threads have a stack size around 1MB. This means that the number of active threads is limited by memory usage. Using a smaller stack allows more threads, but limits the number of stack frames before a stack overflow error. Using dynamically-sized stacks, the number of active goroutines can be much larger.

Channels in Go are essentially thread-safe queue structures. These are ideal for producer-consumer patterns. Enqueue operations on a channel block if the channel is full, and dequeue operations block if the channel is empty.

Within each database node, my system uses a thread confinement pattern to avoid deadlock and race conditions. This is based on a Go idiom: ``Do not communicate by sharing memory; instead, share memory by communicating.'' The state of each node belongs to a single goroutine, and cannot be read or modified directly by any other goroutine \cite{effective}.

Each database node essentially receives messages from the network, and processes them. How it responds to a message depends on a state machine. Based on this, it may send messages to other nodes or clients, change state and modify its local data store (or several or node of those). Importantly, some operations, such as accessing the data store, require waiting for I/O.

The node's local state machine is owned by a single goroutine responsible for the main loop. This loop must not block for a prolonged period time. Other goroutines within the node are responsible for blocking operations, including setting timers. When a transaction times out, for example, the goroutine responsible cannot update the main state machine; instead, it sends a message to the main loop, using a channel. The main loop receives the message and responds appropriately. Importantly, only the main loop can access the state machine, so there can be no race conditions, without using any locks. Care must be taken to ensure the main loop cannot block, but I found this easier in practice than managing locks everywhere.

\section{Out-of-Order Delivery}

Any message sent over a network has some unknown and unbounded latency (really unbounded \cite{imbriaco_2012}). This means that messages may be delivered out-of-order: if a node $A$ sends several messages to node $B$, the order in which they are sent is not always the order in which they arrive. This can cause difficult problems.

What if a node is sent a lock request, then an unlock request very soon after? % TODO: diagram

If the unlock arrives before the lock, the node may become locked forever.

To fix this, each node stores all transaction IDs for which it has received an unlock request. If it subsequently receives a lock request with an already-seen ID, ignore it.

This leads to each node accumulating a record of every transaction ever, so make it soft state.

\section{Remote Procedure Calls}

For distributed systems, in general, clients making requests are not directly connected; they can only communicate using an unreliable network. Given this limitation, some system is needed to encapsulate procedure calls. Such systems are often based on hypertext transfer protocol (HTTP), due to its wide usage on the Internet.

One method is representational state transfer (REST), which is based on the concept of resources. The URL should contain the name of a resource, and the type of operation should be indicated by the HTTP verb (GET, POST, PUT etc.) used. This works well with create, read, update and delete (CRUD) operations which are meaningful in database systems.

One drawback of REST is its inefficiency. Suppose we have 4 types of operations. The information conveyed by specifying the type of operation is 2 bits. The HTTP verb GET uses 3 bytes, so is a very inefficient encoding.

Instead, we can use an RPC library which is not based on text encoding. RPC libraries can also provide significant abstractions, so that RPC calls are very similar to normal function calls, from the client's point of view. For example, Java's remote method invocation system provides an abstraction over serialising and deserialising Java objects, and providing distributed garbage collection.

For this project, I will not use any RPC solution which would be used in a real deployment. Since I am going to run the whole system on a single machine, I can use a much simpler mechanism. HTTP (using transmission control protocol, TCP) provides support for (mostly) reliable communication over unreliable networks, using checksums and re-transmission. On a single machine, I will have a reliable simulated network, so TCP is unnecessary, and would add additional complexity and latency to the system.

In order to evaluate my system, I need to be able to add variable latency and packet loss to simulate different network conditions. By minimising the latency of my RPC service, there will be minimal noise affecting my measurements.

% TODO: I needed to implement some parts of RPC (session management), but Go on a single machine gives me some things for free (presentation)

\section{Test Framework}

My system will be built so that it interfaces closely with my test framework, using my RPC system. This will provide an ideal environment for integration testing and evaluating the system's performance. The architecture is based on two of Go's concurrency primitives: goroutines and channels.

Goroutines are similar to threads. Rather than using OS-level scheduling, the Go runtime has its own scheduler. Goroutines, unlike normal threads, have dynamically-sized stacks, between 8 kB and 1 GB \cite{gosrc}. Typically, threads have a stack size around 1MB. This means that the number of active threads is limited by memory usage. Using a smaller stack allows more threads, but limits the number of stack frames before a stack overflow error. Using dynamically-sized stacks, the number of active goroutines can be much larger.

Goroutines also have more efficient scheduling than threads in the OS. For example, the Go runtime knows which registers are live, so can copy the minimum number of register values, whereas the OS has to assume all registers are live and save the contents of all of them.

Channels in Go are essentially thread-safe queue structures. These are ideal for producer-consumer patterns. Consumers can easily poll the queue, blocking until a value is available. My design uses channels to simulate network links.

Each database node will run as its own goroutine. The system will support a fixed number of nodes, so each node will be identified by a unique integer. These will be assigned sequentially and remain constant, so they can be used as addresses in the simulated network. % TODO: discuss name servers (DNS, alternatives)

For each database node, there will be two channels: one for incoming messages, and one for outgoing messages. A message may be a request from a (simulated) external client, a response to such a request, or a message used to coordinate nodes in the system. A message has 3 fields: source and destination addresses (integers), and a payload, the structure of which depends on the type of message. In order to handle a message, the network simulator only needs to inspect the source and destination addresses.

Supporting each database node, there will be an additional goroutine responsible for delivering its outgoing messages. This will use the following algorithm:

\begin{lstlisting}
while true:
    message = outgoingMessages.poll() // Blocking deque

    // Chance of packet loss, based on a given parameter
    if (uniform random value in range [0,1]) < x:
        continue

    // Random latency
    delay = (normal random value, for given mean and variance)
    sendAfterDelay(message, delay)
\end{lstlisting}

The \verb|sendAfterDelay| method spawns a new goroutine, which sleeps for the given time, then appends the message to the recipient's incoming queue. This means that the main loop will not block, so later messages will be delivered without uncontrolled delay.

The main routine will be responsible for simulating clients. This includes generating requests, sending these requests to random database nodes, and recording responses. For each request, it should record the time from request to response (if a response was received). If a request times out without a response, this should also be recorded.

Once all requests have either timed out or been responded to, the main routine will terminate (thus halting the whole system). By configuring the number and type of tests and recording responses (as well as configuring the network properties, as described above), this can be used for both integration testing and evaluating the system's performance.

% TODO: diagram

\chapter{Evaluation}
% 20% (with conclusions)

General test conditions \cite{nunemaker}, the 8 fallacies % (https://en.wikipedia.org/wiki/Fallacies_of_distributed_computing)

Strict Quorum

- Is the output consistent? Test linearisability \cite{herlihy1990linearizability}

- Performance with difference quorum sizes

- Measured with and without retries, with and without node failures

- Comment on how common node failures are \cite{45855}

- Zoom in on particular test cases to measure performance over time, in detail

- In particular, consider total failure cases

Sloppy Quorum

- Same as above (availability)

- Measure time to convergence (consistency)

What Ifs?

- How are results affected by different transaction sizes?

- What if the distance (latency) between different nodes is different?

- What if we use finer locks (for each key), rather than global locking?

Other Consistency Guarantees?

- Implement and evaluate simple variations from Terry \cite{terry2013}

\chapter{Conclusions}

%%%%%%%%%%%%%%%%%%%%%%%%%%%%%%%%%%%%%%%%%%%%%%%%%%%%%%%%%%%%%%%%%%%%%
% the bibliography
\addcontentsline{toc}{chapter}{Bibliography}
\bibliography{refs}

%%%%%%%%%%%%%%%%%%%%%%%%%%%%%%%%%%%%%%%%%%%%%%%%%%%%%%%%%%%%%%%%%%%%%
% the appendices
\appendix
%
%\chapter{Latex source}
%
%\section{diss.tex}
%{\scriptsize\verbatiminput{diss.tex}}
%
%\section{proposal.tex}
%{\scriptsize\verbatiminput{proposal.tex}}
%
%\chapter{Makefile}
%
%\section{makefile}\label{makefile}
%{\scriptsize\verbatiminput{makefile.txt}}
%
%\section{refs.bib}
%{\scriptsize\verbatiminput{refs.bib}}

\chapter{Project Proposal}
%\addcontentsline{toc}{chapter}{Project Proposal}

\setlength{\parindent}{0em}
\addtolength{\parskip}{1ex}

\vfil

\centerline{\Large Computer Science Tripos Part II Project Proposal}
\vspace{0.4in}
\centerline{\Large A Comparison of Consistency Models in Distributed Database Systems}
\vspace{0.4in}
\centerline{\large A.~T.~Bostock, Churchill College}
\vspace{0.3in}
\centerline{\large Originator: A.~T.~Bostock }
\vspace{0.3in}
\centerline{\large 18 October 2018}

\vfil

\vspace{0.2in}

\noindent
{\bf Project Supervisor:} Dr J.~K.~Fawcett
\vspace{0.2in}

\noindent
{\bf Director of Studies:} Dr J.~K.~Fawcett
\vspace{0.2in}
\noindent
 
\noindent
{\bf Project Overseers:} Prof J.~M.~Bacon, Prof R.~J.~Anderson \& Dr A.~S.~Prorok

\vfil
\pagebreak

% Main document

\section*{Introduction}

Distributed database systems are widely used for internet-scale applications. By splitting vulnerabilities across many nodes, we can build systems that continue to function when some nodes are unavailable.

A key limitation for distributed database systems is the CAP theorem: it is impossible to build a system with all 3 of consistency, availability and partition-tolerance. When designing a system, we have to choose the most important properties.

Given the de-centralised nature of the Internet, internet-scale systems need to be partition-tolerant. This means we have a trade-off between consistency and availability. In a critical system, such as a bank's ledger, consistency is most important. In other cases, it may be desirable to sacrifice strong consistency for better availability.

Consider the design of a social networking platform, which allows users to add posts, and serves those posts to other users. For user-experience, it is probably better to serve whatever posts are immediately available, possibly missing the most recent posts, rather than failing to serve any data. It is probably acceptable to have a few seconds delay before a new post is visible to all users.

In this system, we want eventual consistency, where all updates will be available after some time, but not necessarily immediately. This is a weak guarantee. In reality, we probably want new posts to be completely available in less than 1 minute in the vast majority of cases.

My project will explore how quickly an eventually consistent system can reach consensus, and how resource usage differs between a strongly consistent system and an eventually consistent system.

\section*{Work to Be Done}

I will build a strongly consistent, distributed key-value store based on quorum assembly. The system will store data on a fixed number of nodes, with all data replicated across every node. This will include:

\begin{itemize}
  \item
  Quorum assembly: requiring a minimum number of nodes be involved in every read and write transaction, and each node involved in a transaction be locked. This ensures serialisability of transactions, by preventing any transaction from accessing data concurrently with another transaction modifying the same data.

  \item
  As part of quorum assembly, an atomic commit protocol is needed, to assure that values are either written to a whole write quorum, or not written at all. For this, I will implement the two-phase commit protocol (2PC).

  \item
  To track which nodes have stale data, I will build a vector clock system.

\end{itemize}

The system will be designed in a modular fashion, so that each component can be replaced to make a new version of the system with different characteristics.

I will build, as a module, a sloppy quorum system. Using this, I will build an eventually consistent variant of the database system.

I will build a test framework to test and evaluate the systems. This should simulate a network of nodes using one machine. It should be able to:

\begin{itemize}
  \item
  Simulate an unreliable network, including random latency and packet loss.

  \item
  Simulate failure of individual nodes.

  \item
  Randomly generate streams of requests, and send them to the system. Requests generated should simulate those from a social network application: many read transactions, a smaller (but still large) number of transactions adding a new key (and value), and a small number of transactions modifying or removing an existing key. Requests should be sent randomly, at a given average rate (following a Poisson distribution).

  %\item
  % The system should also be able to send any given sequence of requests. For example, to measure time to reach consistency in an eventually consistent system, I could send a write transaction, and then repeatedly send read transactions to each node, to measure the time to reach consistency.

  \item
  Record data for each request sent: request data, request timestamp, and response data and timestamp (if a response is received).

\end{itemize}

Alongside the metrics reported by my test framework, I can use profiling tools to measure resource usage by different parts of the system.

I will evaluate the strongly-consistent and eventually-consistent variants of the system using some or all of the following metrics:

\begin{itemize}
  \item
  System throughput: average time taken to process a large number of requests.

  \item
  For the eventually consistent version, time taken to reach consistency.

  \item
  CPU and memory usage of each node (and average across all nodes).

\end{itemize}

Each of those can be measured for different numbers of nodes, different quorum (or sloppy quorum) sizes, different rates of transactions, different network conditions, and different scenarios involving node failures.

I will implement the project using Go. Go is a fast, compiled language, making it ideal for low-level systems projects. It is also statically typed, and uses garbage collection, so it is safer and easier to work with than C, a commonly-used language for systems programming. Go also has good support for concurrent programming, which will be helpful for building a test framework, using concurrent components to simulate a distributed system.

\subsection*{Success Criteria}

This project will be considered a success if I have done all of the following:

\begin{itemize}
  \item
  Built a strongly consistent distributed key-value store. This system must, when tested with 5 nodes:

  \begin{itemize}
    \item
    Support get and put operations. \verb|put(key, value)| must either successfully store the given key and value, or error. \verb|get(key)| must either return the most recently put value for the given key, or error.

    \item
    Continue to function properly after 2 nodes have failed.

  \end{itemize}

  \item
  Built a variant of that system which is eventually consistent.

  \item
  Evaluated the throughput of each system when using different numbers of nodes, and different rates of transactions.

  \item
  Evaluated the time taken for a value written to the eventually consistent system to be completely available, under different network conditions.

\end{itemize}

\subsection*{Extensions}

\begin{itemize}
  \item
  Implement three-phase commit (3PC), and evaluate the performance of the system when using 3PC compared to 2PC.

  \item
  Explore and evaluate methods for optimising the sloppy quorum system to minimise the time to reach consistency.

  \item
  Explore different data structures for storing data on each node, and evaluate their performance, including how significant the choice of data structure is compared to the design of the distributed components of the system.

\end{itemize}

\section*{Starting Point}

I have some experience using Go, having previously used it for a personal project (implementing a regular expression processor).

I have previously studied distributed systems in the IB course Concurrent and Distributed Systems.

\section*{Resources Required}

I will use my personal laptop (2017 MacBook Pro). I accept full responsibility for this machine and I have made contingency plans to protect myself against hardware and/or software failure.

 I will use git and GitHub for version control, Backblaze for cloud backup, and an external disk for additional weekly backups.

I will use Go to implement my project, so I will use its open-source compiler and tools.

In case of failure of my machine, I can restore from one of the backups, and continue working by logging onto a remote private server on Digital Ocean (on which I can install Go tools and my preferred editor, Kakoune), using an MCS machine. This should allow me to continue working with minimal interruption.

\section*{Timetable}

\subsection*{18 Oct - 31 Oct}
Produce a detailed design of the system, presented as a document which can be used later when writing the preparation and implementation chapters of the dissertation.

\textbf{Milestone:} Specification document completed, and approved by supervisor.


\subsection*{1 Nov - 14 Nov}
Build a basic test framework, so that I can use it to test the distributed database system.

\textbf{Milestone:} A basic test framework complete, with support for running multiple nodes, generating test transactions, and verifying that system output is as expected.


\subsection*{15 Nov - 28 Nov}
Implement quorum assembly, including 2PC.

\textbf{Milestone:} All tests passing for the quorum assembly module.


\subsection*{29 Nov - 12 Dec}
Build a strongly consistent key-value store, based on quorum assembly.

\textbf{Milestone:} Key-value store implemented, all tests passing.


\subsection*{13 Dec - 26 Dec}
Slack time / work on extensions.

\textbf{Milestone:} All tests passing for the strongly consistent database.


\subsection*{27 Dec - 9 Jan}
No work around Christmas time.

\subsection*{10 Jan - 23 Jan}
Write a progress report.

Prepare a presentation.

\textbf{Milestone:} Progress report submitted (by 12:00 1 February).

\textbf{Milestone:} Presentation prepared.

\subsection*{24 Jan - 6 Feb}

Extend the database system to use a sloppy quorum system, to provide eventual consistency.

\textbf{Milestone:} Presentation delivered.

\textbf{Milestone:} All tests passing for the sloppy quorum version.


\subsection*{7 Feb - 20 Feb}

Extend the testing framework to simulate different conditions and provide system metrics, as required for evaluation.

Evaluate the strongly consistent and eventually consistent versions of the system.

Write a draft evaluation chapter for the dissertation.

\textbf{Milestone:} Draft evaluation chapter written, and sent to supervisor for feedback.


\subsection*{21 Feb - 6 Mar}

Write an outline dissertation, including important figures and a summary of each section's content.

\textbf{Milestone:} Outline dissertation complete, and sent to supervisor for feedback.


\subsection*{7 Mar - 20 Mar}

Slack time / work on extensions.

\textbf{Milestone:} All success criteria met.


\subsection*{21 Mar - 3 Apr}

Write dissertation: preparation and implementation chapters.

\textbf{Milestone:} Preparation and implementation chapters complete, and sent to supervisor for feedback.


\subsection*{4 Apt - 17 Apr}

Write dissertation: all remaining chapters.

Make changes to implementation and evaluation chapters based on feedback.

\textbf{Milestone:} Dissertation complete, and sent to supervisor/DoS for feedback.


\subsection*{18 Apr - 1 May}

Make changes to dissertation based on feedback.

Slack time / work on extensions.

\textbf{Milestone:} Dissertation complete, including changes based on feedback.


\subsection*{2 May - 15 May}

Slack time.

\textbf{Milestone:} Dissertation submitted (by 12:00 17 May).




\end{document}
